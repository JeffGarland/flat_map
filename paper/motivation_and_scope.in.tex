\section{Motivation and Scope}

There has been a strong desire for a more space- and/or runtime-efficient
representation for \code{map} among C++ users for some time now.  This has
motivated discussions among the members of SG14, numerous articles and talks,
and an implementation in Boost, \code{boost::container::flat_map}.  Virtually
everyone who makes games, embedded, or system software in C++ uses the Boost
implementation or one that they rolled themselves.\\

Here are some numbers that show why.  The graphs that follow show runtimes for
different \code{map}-like associative containers.  The containers used are
Boost.FlatMap, \code{std::map}, and two thin wrappers over a sorted
\code{std::vector}; the ``custom pair'' version of the sorted
\code{std::vector} uses a simple \code{struct} instead of \code{std::pair} for
its value type.  All containers use an \code{int} as the key type and an
\code{int} or a \code{struct} with 5 \code{double}s for the value type.\\

These first four graphs cover the \code{int}-value-type case.  The first graph
shows insertion of N elements with random keys; the second shows full
iteration across all N elements; the third shows \code{map.lower_bound()}
called once for each key used in the original insertions; and the fourth shows
erasure of all N elements, by the keys used in the original insertions.

%%% insert / int %%%
%%% iterate / int %%%
%%% lower bound / int %%%
%%% erase / int %%%

These next four graphs are just like the preceding ones, but cover the
\code{struct}-value-type case.

%%% insert / struct %%%
%%% iterate / struct %%%
%%% lower bound / struct %%%
%%% erase / struct %%%

TODO
