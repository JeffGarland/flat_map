\section{Motivation and Scope}

There has been a strong desire for a more space- and/or runtime-efficient
representation for \code{map} among C++ users for some time now.  This has
motivated discussions among the members of SG14 resulting in a
paper\footnote{See P0038R0,
  \href{http://www.open-std.org/jtc1/sc22/wg21/docs/papers/2015/p0038r0.html}{here}.},
numerous articles and talks, and an implementation in Boost,
\code{boost::container::flat_map}\footnote{Part of Boost.Container,
  \href{http://www.boost.org/doc/libs/1_61_0/doc/html/container.html}{here}.}.
Virtually everyone who makes games, embedded, or system software in C++ uses
the Boost implementation or one that they rolled themselves.\\

Here are some numbers that show why.  The graphs that follow show runtimes for
different \code{map}-like associative containers.  The containers used are
Boost.FlatMap, \code{map}, and an implementation of a flat map with separate
\code{vector} storage for keys and values (``split storage'').  All containers
use either \code{<int, int>} or \code{<std::string, std::string>}
for the value type.\\

All data in the graphs below were produced on Windows with MSVC 2017, on
Mac OSX with Clang 4.0 and libc++, or on Linux with g++ 6.2 and libstdc++.\\

Each set of six graphs shows the performance of a single operation on all
map-variants.  The left column shows the \code{<int, int>} runs, and the right
column shows the \code{<std::string, std::string>} ones.  Each row shows one
platform/compiler configuration. \\

These three sets of graphs cover the most commonly-used operations (erasure is
left out, since it is nearly identical to insertion).  The first set shows
insertion of N elements with random keys; the second shows full iteration
across all N elements; and the third shows \code{map.find()} called
once for each key used in the original insertionss.\\

\subsection{Insert}

%%% insert, int, string %%%

Unsurprisingly, insertion takes longer in contiguous-storage implementations.
Boost.FlatMap has the steepest growth curves by far.  Interestingly, the split
storage implementation is roughly halfway in between \code{map} and
Boost.FlatMap for \code{<int, int>} runs.

\subsection{Iterate}

%%% iterate, int, string %%%

For the variants other than \code{map}, iteration is relatively similar, and
much faster than \code{map}'s.

\subsection{Find}

%%% find, int, string %%%

\code{find()} performance is where things get interesting.  The different
platforms produce somewhat similar results.\\

Though the curves look different for GCC, in all cases \code{find()} is
markedly slower for \code{map} than for the flat implementations.


\subsection{Implications}

Iteration is vastly cheaper for contiguous-storage variants.  Any node-based
associative container will always be slower than a flattened one for
iteration.  For use cases where there is a lot of iteration, this can be the
deciding runtime performance consideration.\\

Find operations are also cheaper for congituous-storage variants, though not
as clearly as so as iteration operations.\\

Use cases in which iteration and lookup are much more frequent than insertion
and deletion suggest the use of a flat implementation.\\

Use cases in which the runtime performance of a flat map would be no better
than \code{map} or \code{unordered_map}, the user may still decide to use a
flat implementation for the storage savings.
