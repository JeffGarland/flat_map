\section{Motivation and Scope}

There has been a strong desire for a more space- and/or runtime-efficient
representation for \code{map} among C++ users for some time now.  This has
motivated discussions among the members of SG14 resulting in a
paper\footnote{See P0038R0,
  \href{http://www.open-std.org/jtc1/sc22/wg21/docs/papers/2015/p0038r0.html}{here}.},
numerous articles and talks, and an implementation in Boost,
\code{boost::container::flat_map}\footnote{Part of Boost.Container,
  \href{http://www.boost.org/doc/libs/1_61_0/doc/html/container.html}{here}.}.
Virtually everyone who makes games, embedded, or system software in C++ uses
the Boost implementation or one that they rolled themselves.\\

Here are some numbers that show why.  The graphs that follow show runtimes for
different \code{map}-like associative containers.  The containers used are
Boost.FlatMap, \code{map}, and two thin wrappers over a sorted \code{vector};
the ``custom pair'' version of the sorted \code{vector} uses a simple
\code{struct} instead of \code{pair} for its value type.  All containers use
an \code{int} as the key type and an \code{int} or a \code{struct} with 5
\code{double}s for the value type.\\

All the graphs below were produced on Windows with MSVC 2015.  Similar results
were obtained on Linux, with Clang 3.9 and libc++, and with g++ 4.8.4 and
libstdc++.\\

These four TODO graphs cover the \code{int}-value-type case.  The first graph
shows insertion of N elements with random keys; the second shows full
iteration across all N elements; the third shows \code{map.find()} called once
for each key used in the original insertions; and the fourth shows erasure of
all N elements, by the keys used in the original insertions.

%%% insert, int %%%
%%% insert, string %%%

As one might expect, insertionion takes longer in contiguous-storage
implementations.  Boost.FlatMap and a sorted \code{vector<pair<int, int>>}
have superlinear growth in insertion time.  While the curve for sorted
\code{vector} using a custom \code{struct} instead of a \code{pair} is
superlinear as well, it is dramatically flatter in its growth -- much closer
to node-based \code{map}.

%%% iterate, int %%%
%%% iterate, string %%%

For all variants but \code{map}, iteration is relatively similar, and much
faster that \code{map}'s.

%%% find, int %%%
%%% find, string %%%

\code{find()} performance is roughly similar across all the
implementations, and they all show superlinear growth.  Note that
Boost.FlatMap performs the best here.

%%% erase, int %%%
%%% erase, string %%%

Erasure has a similar performance profile to insertion, except that the sorted
\code{vector<pair<int, int>>} performs substantially better than
Boost.FlatMap.\\


\subsection{Implications}

TODO Iteration is vastly cheaper for contiguous-storage variants.  It has been
suggested that a \code{map} with a custom allocator can achieve similar
performance to flat data structures, but this would not apply to iteration
performance, unless the values were added to the \code{map} in sorted order.\\

In all the graphs above, the reason the custom-\code{pair} sorted vector
performs so much better than \code{vector<pair<int, int>>} seems to be that
the custom-\code{pair} type has \code{nothrow} special functions.
Implementing all the special functions and adding \code{nothrow(false)} to
each makes the custom-\code{pair} version perform identically to the
\code{pair<int, int>} version.

Boost.FlatMap differs quite a bit from a sorted \code{vector}.  Clearly there
are a lot of QOI choices to make in implementing a standard \code{flat_map}.
