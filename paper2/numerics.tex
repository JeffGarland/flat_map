\rSec0[std2.numerics]{Numerics library}

\synopsis{Header \tcode{<\changed{experimental/ranges}{std2}/random>} synopsis}

\begin{codeblock}
namespace @\changed{std \{ namespace experimental \{ namespace ranges}{std2}@ { inline namespace v1 {
  template <class G>
  concept @\removed{bool}@ UniformRandomNumberGenerator = @\seebelow@;
}}@\removed{\}\}}@
\end{codeblock}

%%%%%%%%%%%%%%%%%%%%%%%%%%%%%%%%%%%%%%%%%%%%%%%%%%%%%%%%%%%%%%%%%%%%%%
% Uniform Random Number Generator requirements:

\rSec1[std2.rand.req.urng]{Uniform random number generator requirements}%
\indextext{uniform random number generator!requirements|(}%
\indextext{requirements!uniform random number generator|(}

\begin{codeblock}
template <class G>
concept @\removed{bool}@ UniformRandomNumberGenerator =
  Invocable<G&> &&
  UnsignedIntegral<result_of_t<G&()>> &&
  requires {
    { G::min() } -> Same<result_of_t<G&()>>&&;
    { G::max() } -> Same<result_of_t<G&()>>&&;
  };
\end{codeblock}

\pnum
A \techterm{uniform random number generator}
\tcode{g} of type \tcode{G}
is a function object
returning unsigned integer values
such that each value
in the range of possible results
has (ideally) equal probability
of being returned.
\enternote
 The degree to which \tcode{g}'s results
 approximate the ideal
 is often determined statistically.
\exitnote

\pnum
Let \tcode{g} be any object of type \tcode{G}.
\tcode{UniformRandomNumberGenerator<G>} is satisfied only if

\begin{itemize}
\item Both \tcode{G::min()} and \tcode{G::max()} are constant expressions~(\cxxref{expr.const}).
\item \tcode{G::min() < G::max()}.
\item \tcode{G::min() <= g()}.
\item \tcode{g() <= G::max()}.
\item \tcode{g()} has amortized constant complexity.
\end{itemize}
